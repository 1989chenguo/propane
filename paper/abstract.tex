\textbf{Abstract---}
Networks are notoriously difficult to configure correctly. 
Operators typically rely on local configuration of individual devices running 
distributed control plane protocols to achieve network-wide objectives. 
Managing implicit dependencies between device configurations and properly accounting for failures makes 
a network operator's work particularly difficult. 
%
The recent move towards software-defined networking has been motivated, in part, 
by the simplicity of the centralized model, where a controller directly manages 
individual devices rather than letting them manage themselves. However distributed 
protocols have many important advantages in scalability,
latency and fault tolerance over the completely centralized approach.
%
In this paper, we describe a new system -- \sysname, which allows an operator to specify a high-level, 
centralized routing policy declaratively, but then to compile it to a purely distributed 
implementation, which uses the BGP routing protocol to run on existing, commodity hardware. 
%
Propane allows operators to describe the types of paths traffic may (or may not) take 
through the network, as well as relative preferences (or backups) between paths. 
%
Network policies compiled with \sysname are guaranteed to implement the correct routing policy regardless of the number of failures, if possible.  If this is not possible, the compiler notifies the user at compile time rather than as the network is in 
operation.
%
%\sysname also takes the centralized model one step further by unifying inter and intra-domain routing. Operators can specify routing policy for an autonomous system by programming against the abstraction that they can control any route in the wider internet.
%
We have implemented a compiler for \sysname and evaluated it against several real-world configurations for data centers and core backbone networks. We demonstrate the \sysname compiler can scale to 
topologies with thousands of routers.
