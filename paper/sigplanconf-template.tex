%-----------------------------------------------------------------------------
%
%               Template for sigplanconf LaTeX Class
%
% Name:         sigplanconf-template.tex
%
% Purpose:      A template for sigplanconf.cls, which is a LaTeX 2e class
%               file for SIGPLAN conference proceedings.
%
% Guide:        Refer to "Author's Guide to the ACM SIGPLAN Class,"
%               sigplanconf-guide.pdf
%
% Author:       Paul C. Anagnostopoulos
%               Windfall Software
%               978 371-2316
%               paul@windfall.com
%
% Created:      15 February 2005
%
%-----------------------------------------------------------------------------


\documentclass{sigplanconf}

% The following \documentclass options may be useful:

% preprint      Remove this option only once the paper is in final form.
% 10pt          To set in 10-point type instead of 9-point.
% 11pt          To set in 11-point type instead of 9-point.
% authoryear    To obtain author/year citation style instead of numeric.

\usepackage{amsmath}


\begin{document}

\special{papersize=8.5in,11in}
\setlength{\pdfpageheight}{\paperheight}
\setlength{\pdfpagewidth}{\paperwidth}

\conferenceinfo{CONF 'yy}{Month d--d, 20yy, City, ST, Country} 
\copyrightyear{20yy} 
\copyrightdata{978-1-nnnn-nnnn-n/yy/mm} 
\doi{nnnnnnn.nnnnnnn}

% Uncomment one of the following two, if you are not going for the 
% traditional copyright transfer agreement.

%\exclusivelicense                % ACM gets exclusive license to publish, 
                                  % you retain copyright

%\permissiontopublish             % ACM gets nonexclusive license to publish
                                  % (paid open-access papers, 
                                  % short abstracts)

\titlebanner{banner above paper title}        % These are ignored unless
\preprintfooter{short description of paper}   % 'preprint' option specified.

\title{Title Text}
\subtitle{Subtitle Text, if any}

\authorinfo{Name1}
           {Affiliation1}
           {Email1}
\authorinfo{Name2\and Name3}
           {Affiliation2/3}
           {Email2/3}

\maketitle

\begin{abstract}
This is the text of the abstract.
\end{abstract}

\category{CR-number}{subcategory}{third-level}

% general terms are not compulsory anymore, 
% you may leave them out
\terms
term1, term2

\keywords
keyword1, keyword2

\section{Introduction}


%% Word vommit
\begin{itemize}
	\item Network configuration is hard ...
	\item Configs are local and distributed
	\item Difficult to express intent directly
	\item Often achieve intent through indirect means (i.e., backups through aggregation)
	\item Distributed, asynchronous, environment unknown, failures ... means difficult to consider all scenarios	
	\item Problems are compounded in inter-domain routing where other factors like peering agreements are important
	\item Wide variety of considerations: Have preferences for certain routes over others, failure resistance, backups etc
	\item On the other hand, BGP is the only widely used, policy-rich routing language
	\item Often used for intra-domain routing as well
	\item We present a new language that allows operators to directly express intent for inter and intra-domain routing policies 
		  from a centralized point of view.
	\item The language provides the operator with the abstraction that they can control all end-to-end routes that transit their 
		  network, even if the routes start and/or end outside of their network.
	\item The language consists of regular expression constraints on the set of allowed routes along with a flexible preference operator
		  that allows ranking sets of paths to enable both inter-domain routing preferences and intra-domain backup paths.
	\item A compiler takes a routing policy written in this language and synthesizes concrete, per-device configurations running BGP. 
	\item The resulting configurations are guaranteed to correctly implement the policy using the standard, distributed path vector protocol, 
		  and remain correct under all possible network failure scenarios.
	\item To enable a wider spectrum of implementable policies, we model common environment assumptions (e.g., semantics of BGP community values)
		  As sets of allowed/disallowed paths provided by a neighbor using the same regular path framework.
	\item Although we choose to target BGP for both inter and intra domain routing due to its expressiveness, our language is agnostic to any particular 
		  routing protocol. It describes only routes and their relative preferences. As such, an alternative routing protocol, such as OSPF, could 
		  be used internally. This would change the set of implementable policies.
	\item To allow for more fine-grained control over control-plane details, our language permits additional constraints on the control protocol
		  (e.g., maximum number of allowed routes, whether to use multipath etc)
	\item Advantages: Centralized configuration (no copy paste error, more concise), Synthesis (no need for gory details, e.g., community values),
		  Expressiveness (backup routing, preferences, waypointing, directly say things like no transit), 
		  Correctness: (Guaranteed to implement the policy, even under failure scenarios),
		  Verification: (Language serves as a high-level abstraction of the low-level details, so we can perform verification directly on the high-level policies)
\end{itemize}



\section{Overview}

\begin{itemize}
	\item Introduce language with examples...
	\item Maybe, start with a typical AS, and show how simple it would be to implement a typical policy
	\item Another good example, would be something that combined internal backup routing with preference based 
		  entrance into the network using aggregation.
	\item Could probably show an example of the resulting config, and how it is tagging using communities, 
		  or alternatively show that it doesn't compile due to a particular failure scenario.
\end{itemize}

\section{Language} 

\subsection{Syntax}
Define the grammar of the language.
Need to distinguish whether the preference operator 
appears only at the top level or not.

\subsection{Semantics} 
Give the semantics in terms of ranked paths.
Should be straightforward extension of regular expression semantics with the 
additional constraint that we take only valid topology paths and that their are no paths with 
repeated locations in the path. If preferences appear only at the top level, then it is easy to rank the paths.
One tricky point is that we care about paths starting or ending at locations that are connected to end-hosts in some way.
And we want the length of the path to be a secondary tie-breaker .
Additional assumption here should be that the prefixes are disjoint.

\subsection{Theory}
What are the properties of this language? Ideally, a proof here about its expressiveness. We can perhaps do everything 
BGP can do? We can't do shortest paths routing with BGP, but we could with OSPF though? Clearly we can only describe 
policies that are stable under all possible failure scenarios (at least internally stable, could still cause) routing 
loops with interactions with neighbors.


\section{Compilation}

\begin{itemize}
	\item Separate prefixes so they are disjoint (assumption) for semantics)
		  probably using BDDs where each bit is a variable is the easiest way to do this.
	\item For each prefix, build a DFA from each of the regular path constraints 
		  We do this using character classes and regular expression derivatives.
		  The accepting states are labeled with the particular preference.
	\item Build the constraint graph, which gives us valid paths through the network that satisfy one or 
		  more of the regular path constraints. We accept if we are in at least one of the accepting states 
		  for one or more of the DFAs. We take the max of the preferences for the accepting preference.
		  Add a new start and end node as well.
		  
	\item Minimize the constraint graph by removing nodes and edges that never contribute to a solution. 
		  This is sound. Removing these makes it easier to check if things are implementable using BGP (see next point).
		  There are two ways to do this: a sound heuristic for removing edges and nodes, and a complete, but slow method for 
		  removing everything not needed.
		  
	\item Since our policy language can express things not implementable in BGP, we need a few checks.
		  First show some examples of what the problem is: why BGP cannot make a decision. 
		  The reasons this can happen are: 
		  (1) Single best path per router - no subset of paths specified is consistent (if A goes through B and C goes through A, then C goes through B).
		  (2) Failures - Have to make a decision without knowing what failures have occurred. Must be the case that 
		                 

	\item We can check for topology consistency using a sufficient condition (that works often and is very fast), 
		  and by enumerating failure scenarios using a necessary and sufficient condition (that always works but is slooooow)
	
	\item Finally, if we know BGP can make decision locally, then we can directly translate to local BGP configurations using 
		  information about the state machines from the neighbors. In particular, we tag (and test on) community values for the
		  corresponding state and set preferences accordingly (example will be needed)
	\item Running example will be helpful here.	
\end{itemize}

\subsection{Optimizations}

\begin{itemize}
	\item Compress communities by reusing communities across routers.
	      That is, we can reuse the same community value since we know who the next hop is 
	      (e.g., if there are 4 Y nodes, then use communities 0-4 for Y)

	\item If we have the same match across all starting locations, then match only on community.
		  If we have the same match across all communities (per peer) for a location, then omit checking 
		  the community (similarly avoid tagging the community from the other point of view)
		  These mainly just make things much easier to read for a human (more natural policy).
\end{itemize}



\section{Evaluation}

\begin{itemize}
	\item Expressiveness of the policy language (maybe a proof earlier for this)
	\item Compactness of the policy language (how verbose? likely pretty concise)
	\item Intuitiveness of the policy language (how easy is it to understand? probably make the earlier examples clear)
	\item Speed of compilation (does it take forever?)
	\item Understandability of the resulting configurations (Can humans understand them?)
	\item Some measure of the size of the resulting configurations (is this actually an issue in BGP??)
\end{itemize}

\section{Related Work}

\begin{itemize}
	\item Regular path queries
	\item Lots of background on BGP
	\item Feamster rcc stuff
	\item Logic programming BGP configs
\end{itemize}

\section{Conclusions}

TODO

\appendix
\section{Appendix Title}

This is the text of the appendix, if you need one.

\acks

Acknowledgments, if needed.

% We recommend abbrvnat bibliography style.

\bibliographystyle{abbrvnat}

% The bibliography should be embedded for final submission.

\begin{thebibliography}{}
\softraggedright

\bibitem[Smith et~al.(2009)Smith, Jones]{smith02}
P. Q. Smith, and X. Y. Jones. ...reference text...

\end{thebibliography}


\end{document}

%                       Revision History
%                       -------- -------
%  Date         Person  Ver.    Change
%  ----         ------  ----    ------

%  2013.06.29   TU      0.1--4  comments on permission/copyright notices

