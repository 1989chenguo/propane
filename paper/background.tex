\section{Background on BGP}
\label{sec:background}

%Our work focuses on BGP as it is the most common and flexible way to implement distributed control planes today. In this section, we provide relevant background on BGP.

BGP is a path-vector routing protocol that connects autonomous systems (ASes). An AS has one or more routers managed by the same administrative entity. ASes exchange routing announcements with their neighbors. Each announcement has a destination IP prefix and some attributes (see below), and it indicates that the sending AS is willing to carry traffic destined to that prefix from the receiving AS. Traffic flows in the opposite direction, from announcement receivers to senders.

When a route announcement is received by an AS, it is processed by custom import filters that may drop the announcement or modify some attributes. If multiple announcements for the same prefix survive import filters, the router selects the best one based on local policy (see below). This route is then used to send traffic to the destination. It is also advertised to the neighbors, after processing through neighbor-specific export filters that may stop the announcement or modify some attributes.

All routing announcements are accompanied by an AS-path attribute that reflects the sequence of ASes that the announcement has traversed thus far. While the AS-path attribute has a global meaning, some attributes are meaningful only within an AS or between neighboring ASes.  One such attribute is a list of community strings. ASes use such strings to color routes on different criteria (e.g., ``entered on West Coast'') and then use the color later in the routing process.  Communities are also used to signal to neighbors how they should handle an announcement (e.g., do not export it further). Another non-global attribute is the multi-exit discriminator (MED). It is used when an AS has multiple links to a neighboring AS.  Its (numeric) values signal to the neighbor how this AS prefers to receive traffic among those links.

The route selection process assigns a {\em local preference} to each
route that survives the import filters. Routes with higher local
preference are preferred. Among routes with the same local preference,
other factors such as AS path length, MEDs, and internal routing cost, are considered in order. Because it is considered first during route selection, local preference is highly influential, and ASes may assign this preference based on any aspect of the route. A common practice is to assign it based on the commercial relationship with the neighbor. For instance, an AS may prefer in order customer ASes (which pay money), peer ASes (with free exchange of traffic), and provider ASes (which charge money for traffic).

In implementing their policy, each network operator assumes that
neighboring ASes correctly implement BGP and honor contracts related
to MEDs and communities. \sysname makes the same assumption when
deriving BGP configurations for a network.

The combination of arbitrary import and export filters and route selection policies at individual routers make BGP a highly flexible routing protocol. That flexibility, however, comes at the cost of it being difficult to configure correctly, as we explain next.
